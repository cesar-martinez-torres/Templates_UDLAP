% Realizada por César Martínez cesar.martinez@udlap.mx
%VERSION 2.0


\documentclass[12pt]{article}  %tipo de documento y tamaño de letra normal
%%%%%%%%%%%%%%%%%%%%%%%%%%%%%%%%%%%%%%%%%%%%%%%%%%%%%%%%%%%%%%%%%%%%%
%%%%%%%%%%%%%%%%%%%%%%%%%%%%%%%%%%%%%%%%%%%%%%%%%%%%%%%%%%%%%%%%%%%%%
%%%%%%%%%%%%%%%%%%%%%%%%%%%%%%%%%%%%%%%%%%%%%%%%%%%%%%%%%%%%%%%%%%%%%
%%%%%%%% Paquetes basicos, pueden encontrar información especifica de cada uno de ellos en  (https://www.ctan.org/) %%%%%%%%%%%%%%%%%%%%%%% %%%%%%%%%%%%%%%%%%%%%%%%%%%%%%%%%%%%%%%%%%%%%%%%%%%%%%%%%%%%%%%%%%%%%
%%%%%%%%%%%%%%%%%%%%%%%%%%%%%%%%%%%%%%%%%%%%%%%%%%%%%%%%%%%%%%%%%%%%%
%%%%%%%%%%%%%%%%%%%%%%%%%%%%%%%%%%%%%%%%%%%%%%%%%%%%%%%%%%%%%%%%%%%%%
\usepackage[spanish]{babel} %Indica que escribiermos en español
%\usepackage[english]{babel} %Indica que escribiermos en inglés
%Comentar la línea del idiona que NO usarán en su reporte
\usepackage[utf8]{inputenc} %Indica qué codificación se está usando ISO-8859-1(latin1)  o utf8  
\usepackage{amsmath} % Comandos extras para matemáticas (cajas para ecuaciones,etc)
\usepackage{amssymb} % Simbolos matematicos (por lo tanto)
\usepackage{graphicx} % Incluir imágenes en LaTeX
\usepackage{color} % Para colorear texto
\usepackage{subfigure} % subfiguras
\usepackage{enumerate} % enumerar
\usepackage{commath} % funcionalidades extras para diferenciales, integrales,etc (\od, \dif, etc)
\usepackage{cancel} % para cancelar expresiones (\cancelto{0}{x})
\usepackage{float} %Podemos usar el especificador [H] en las figuras para que se queden donde queramos
\usepackage{appendix} %Para crear apendices
\usepackage{xcolor} %Definir colores personalizados
%%%%%%%%%%%%%%%%%%%%%%%%%%%%%%%%%%%%%%%%%%%%%%%%%%%%%%%%%%%%%%%%%%%%%
%%%%%%%%% PAQUETES CON OPCIONES ESPECIFICAS PRECARGADAS %%%%%%%%%%%%%
%%%%%%%%%%%%%%%%%%%%%%%%%%%%%%%%%%%%%%%%%%%%%%%%%%%%%%%%%%%%%%%%%%%%%
%%%%%%%%%%%%% Permitir agregar código, colocarlo en un rectángulo y  numerarlo %%%%%%%%%%%%%%%%%%%%%%%%%%%%%%%%%%%%%%%%%%%%%%%%%%%%%%%%%%%
%%%%%%%%%%%%%%%%%%%%%%%%%%%%%%%%%%%%%%%%%%%%%%%%%%%%%%%%%%%%%%%%%%%%%
%%%%%%%%%%%%%%%%%%%%%%%%%%%%%%%%%%%%%%%%%%%%%%%%%%%%%%%%%%%%%%%%%%%%%%%%%%%%%%%%%%%%%%%%%%%%%%%%%%%%%%%%%%%%%%%%%%%%%%%%%%%%%%%%%%%%%%%%%%
\usepackage{listings} %Sirve para pegar codigo fuente de programas
\usepackage{caption} %Agregar titulos a los codigos
\DeclareCaptionFont{white}{\color{white}}
\DeclareCaptionFormat{listing}{%
  \parbox{\textwidth}{\colorbox{gray}{\parbox{\textwidth}{#1#2#3}}\vskip-4pt}}
\captionsetup[lstlisting]{format=listing,labelfont=white,textfont=white}
\lstset{frame=lrb,xleftmargin=\fboxsep,xrightmargin=-\fboxsep}
\renewcommand{\lstlistingname}{Código}
%%%%%%%%%%%%%%%%%%%%%%%%%%%%%%%%%%%%%%%%%%%%%%%%%%%%%%%%%%%%%%%%%%%%%
%%% Definir márgenes del documento%%%%%%%%%%%%%%%%%%%%%%%%%%%%%%%%%%%
%%%%%%%%%%%%%%%%%%%%%%%%%%%%%%%%%%%%%%%%%%%%%%%%%%%%%%%%%%%%%%%%%%%%%
 \usepackage{anysize} % Para personalizar el ancho de  los márgenes
\marginsize{2cm}{2cm}{2cm}{2cm} % Izquierda, derecha, arriba, abajo
%%%%%%%%%%%%%%%%%%%%%%%%%%%%%%%%%%%%%%%%%%%%%%%%%%%%%%%%%%%%%%%%%%%%
%%% Hipervinculos activos y a color %%%%%%%%%%%%%%%%%%%%%%%%%%%%%%%%
%%%%%%%%%%%%%%%%%%%%%%%%%%%%%%%%%%%%%%%%%%%%%%%%%%%%%%%%%%%%%%%%%%%%%
\usepackage[colorlinks=true,plainpages=true,citecolor=blue,linkcolor=black]{hyperref}
\usepackage{hyperref} 
%%%%%%%%%%%%%%%%%%%%%%%%%%%%%%%%%%%%%%%%%%%%%%%%%%%%%%%%%%%%%%%%%%%%%
%%%%%% Encabezado y pie de pagina %%%%%%%%%%%%%%%%%%%%%%%%%%%%%%%%%%%
%%%%%%%%%%%%%%%%%%%%%%%%%%%%%%%%%%%%%%%%%%%%%%%%%%%%%%%%%%%%%%%%%%%%%
\usepackage{fancyhdr} 
\pagestyle{fancy}
\fancyhf{}
\fancyhead[L]{\footnotesize UDLAP} %encabezado izquierda
\fancyhead[R]{\footnotesize CEM}   % encabezado derecha
\fancyfoot[R]{\footnotesize \curso}  % Pie derecha
\fancyfoot[C]{\thepage}  % centro
\fancyfoot[L]{}  %izquierda
\renewcommand{\footrulewidth}{0.4pt}
%%%%%%%%%%%%%%%%%%%%%%%%%%%%%%%%%%%%%%%%%%%%%%%%%%%%%%%%%%%%%%%%%%%%%
%%%% Carpeta donde se deben colocar las imagenes %%%%%%%%%%%%%%%%%%%%
\graphicspath{{Imagenes/}} %Colocar aqui todas las imagenes del documento pueden estar en formato png, eps o jpg, se recomienda eps para mayor calidad.
%%%%%%%%%%%%%%%%%%%%%%%%%%%%%%%%%%%%%%%%%%%%%%%%%%%%%%%%%%%%%%%%%%%%%
%%%%%%%%%%%%%%%%%%%%%%%%%%%%%%%%%%%%%%%%%%%%%%%%%%%%%%%%%%%%%%%%%%%%%
%%%%%%%% Termina carga de paquetes %%%%%%%%%%%%%%%%%%%%%%%%%%%%%%%%%%
%%%%%%%%%%%%%%%%%%%%%%%%%%%%%%%%%%%%%%%%%%%%%%%%%%%%%%%%%%%%%%%%%%%%%
%%%%%%%%%%%%%%%%%%%%%%%%%%%%%%%%%%%%%%%%%%%%%%%%%%%%%%%%%%%%%%%%%%%%%
%%%%%%%%%%%%%%%%%%%%%%%%%%%%%%%%%%%%%%%%%%%%%%%%%%%%%%%%%%%%%%%%%%%%%
%%%%%% Modificar campos que aparecerán en portada %%%%%%%%%%%%%%%%%%%
%%%%%%%%%%%%%%%%%%%%%%%%%%%%%%%%%%%%%%%%%%%%%%%%%%%%%%%%%%%%%%%%%%%%%
%%%%%%%%%%%%%%%%%%%%%%%%%%%%%%%%%%%%%%%%%%%%%%%%%%%%%%%%%%%%%%%%%%%%%
%%%%%%%%%%%%%%%%%%%%%%%%%%%%%%%%%%%%%%%%%%%%%%%%%%%%%%%%%%%%%%%%%%%%%
\def\titulo{Reporte final}%titulo del documento
\def\materia{Clave materia sección: } %Clave nombre de la materia y sección
\def\curso{materia} %Nombre de la materia para footnote
\def\fecha{Fecha de entrega} %En formato "dia" de "mes" de "año"
\def\equipo {Número de equipo}%Verificar en blackboard el número asignado
\def\ida{Id del estudiante 1} %Nombre y Id´s de todos los integrantes que hayan trabajado en el proyecto
\def\esta{Nombre 1}
\def\idb{Id del estudiante 2}
\def\estb{Nombre 2}
\def\idc{Id del estudiante 3}
\def\estc{Nombre 3}
\def\idd{Id del estudiante 4}
\def\estd{Nombre 4}
\def\ide{Id del estudiante 5} 
\def\este{Nombre 5}%Copiar y pegar más líneas si su equipo tiene más de 5 integrantes, eliminar si está formado por menos
%%%%%%%%%%%%%%%%%%%%%%%%%%%%%%%%%%%%%%%%%%%%%%%%%%%%%%%%%%%%%%%%%%%%%
%%%%%%%%%%%%%%%%%%%%%%%%%%%%%%%%%%%%%%%%%%%%%%%%%%%%%%%%%%%%%%%%%%%%%
%%%%%%%%%%%%%%%%%%%%%%%%%%%%%%%%%%%%%%%%%%%%%%%%%%%%%%%%%%%%%%%%%%%%%
\begin{document} %Inicia el documento
%%%%%%%%%%%%%%%%%%%%%%%%%%%%%%%%%%%%%%%%%%%%%%%%%%%%%%%%%%%%%%%%%%%%%
%%%%%%%%%%%%%%%%%%%%%%%%%%%%%%%%%%%%%%%%%%%%%%%%%%%%%%%%%%%%%%%%%%%%%
%%%%%%%%%%%%%%%%%%%%%%%%%%%%%%%%%%%%%%%%%%%%%%%%%%%%%%%%%%%%%%%%%%%%%
%%%%%%%%%%%%%%%%%%%%%%%%%%%%%%%%%% PORTADA %%%%%%%%%%%%%%%%%%%%%%%%%%
%%%%%%%%%%%%%%%%%%%%%%%%%%%%%%%%%%%%%%%%%%%%%%%%%%%%%%%%%%%%%%%%%%%%%No es necesario modificar ninguna de las siguientes lineas, sólo si el número de estudiantes que conforman su equipo es menor o mayor a 5
%%%%%%%%%%%%%%%%%%%%%%%%%%%%%%%%%%%%%%%%%%%%%%%%%%%%%%%%%%%%%%%%%%%%%
%%%%%%%%%%%%%%%%%%%%%%%%%%%%%%%%%%%%%%%%%%%%%%%%%%%%%%%%%%%%%%%%%%%%%
%%%%%%%%%%%%%%%%%%%%%%%%%%%%%%%%%%%%%%%%%%%%%%%%%%%%%%%%%%%%%%%%%%%%%
\begin{center}														
\newcommand{\HRule}{\rule{\linewidth}{0.5mm}}						
\thispagestyle{empty} 												
\vspace*{-1.5cm}								
\textsc{\huge Universidad de las Américas Puebla}\\[1.5cm]	
\textsc{\LARGE Escuela de ingeniería}\\[1.5cm]	
\textsc{\LARGE Departamento de computación, electrónica y mecatrónica}\\[1.5cm]												
\includegraphics[width=150mm]{UDLAP}  									\vspace*{1cm}														\HRule \\[0.4cm]												
{ \huge \bfseries \titulo}\\[0.4cm]	
\HRule \\[1cm]														
{ \Large \bfseries \materia}\\[1cm] 	
{ \Large \bfseries Equipo \equipo}\\[1cm] 							
\begin{flushleft} \Large											
\ida \hspace{0.5cm}\esta \\
\idb \hspace{0.5cm}\estb \\
\idc \hspace{0.5cm}\estc \\
\idd \hspace{0.5cm}\estd \\
\ide \hspace{0.5cm}\este \\ %Copiar y pegar más líneas si su equipo tiene más de 5 integrantes, eliminar si está formado por menos
\end{flushleft}														
\vfill																
\begin{center}													
{\Large A \fecha, San Andrés Cholula, Puebla}						
\end{center}												 		
\end{center}							 								\newpage						
%%%%%%%%%%%%%%%%%%%% TERMINA PORTADA %%%%%%%%%%%%%%%%%%%%%%%%%%%%%%%%
%%%%%%%%%%%%%%%%%%%%%%%%%%%%%%%%%%%%%%%%%%%%%%%%%%%%%%%%%%%%%%%%%%%%%
%%%%%%%%%%%%%%%%%%%%%%%%%%%%%%%%%%%%%%%%%%%%%%%%%%%%%%%%%%%%%%%%%%%%%
%%%%%%%%%%%%%%%%%%%%%%%%%%%%%%%%%%%%%%%%%%%%%%%%%%%%%%%%%%%%%%%%%%%%%
%%%%%%%%%%%%%%%%%%%%%%%%%%%%%%%%%%%%%%%%%%%%%%%%%%%%%%%%%%%%%%%%%%%%%
\thispagestyle{empty} %Para no numerar la página del indice
\tableofcontents %Comando que genera el índice
\newpage %Se asegura que el documento inicie en la sigueinte pagina despues del índice
\setcounter{page}{1} %Para comenzar a numerar las páginas desde este punto
\section{Sección}
\cite{gh1562} %Citar una referencia

$\cancelto{0}{x}$ %Camcelar x y colocar un cero
\begin{itemize}
    \item Item 1
    \item Item 2
    \item Item 3
    \item Item 4
\end{itemize}

\subsection{Subsección}
\cite{Agarwal2019}
\subsubsection{Sub-subsección}
%% NOTA Aparecern warnings cada vez que se agregue código, ignorar por el momenoto, la siguiente versión del template corregirá estos errores

% En este caso será necesario copiar y pegar todo el código 
\begin{lstlisting}[language=Matlab, caption=Código de matlab]
  %init_quad_reconfig_add.m %Ejecutar el programa de inicializacion
%%
clear all
close all
m=0.52%-0.0520; %masa dron
g=9.81;
tf=150;% Numero de muestras de la simulacion

\end{lstlisting}

%En este caso el código puede ser mandado a llamar desde un archivo
\lstinputlisting[language=C,caption=Código desde archivo y en C]{codigos/fast.c}


% En este caso será necesario copiar y pegar todo el código 
\begin{lstlisting}[language=VHDL, caption=Código en VHDL]
  library IEEE;
use IEEE.STD_LOGIC_1164.ALL;
use IEEE.std_logic_unsigned.all;
entity Contador0_9 is
    Port ( clock_100Mhz : in STD_LOGIC;
           reset : in STD_LOGIC;
           Anode_Activate : out STD_LOGIC_VECTOR (3 downto 0);
           LED_out : out STD_LOGIC_VECTOR (6 downto 0);
           Led0: out STD_LOGIC:='0';
           Led1: out std_logic:='0');-- COMMENT
end Contador0_9;
\end{lstlisting}

% En este caso será necesario copiar y pegar todo el código 
\begin{lstlisting}[language=Python, caption=Código en Python]
  def hello(name):
    print 'Hello', name
 
if __name__=='__main__':
    hello('Me')   #COMENTARIO 
\end{lstlisting}


\newpage %Termina la pagina y empieza una nueva
\appendix % A partir de este comando, todas las secciones por venir serán parte del apéndice
\section{Hoja de software}


%%%%%%% Bibliografía %%%%%%%%
\clearpage %Asegura que la bibliografía inicie en una nueva página
\bibliographystyle{bst/IEEEtran} %Estilo de bibliografía
\bibliography{bib/bibliografia} %Fuentes bibliográficas
\addcontentsline{toc}{section}{Referencias}  %Agrega la bilbiogrfía al indice
%%%%%%% Bibliografía %%%%%%%%      
\end{document} %Termina el documento