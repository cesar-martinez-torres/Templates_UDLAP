\documentclass[fontsize=11pt, parskip=half]{scrartcl}
\usepackage[utf8]{inputenc}
\usepackage[T1]{fontenc}
\usepackage{textcomp}
\usepackage{graphicx}
\usepackage{enumitem}
\graphicspath{{Figures/}} 
\setlength{\parindent}{0em}
% set section in CM
\setkomafont{section}{\normalfont\bfseries\Large}
\newcommand{\mytitle}[1]{{\noindent\Large\textbf{#1}}}
\newcommand{\mysection}[1]{\textbf{\section*{#1}}}
%%%%%% Documento es español %%%%%%
\newcommand{\question}[1]{\textbf{Pregunta:} #1}
\newcommand{\answer}{\textbf{Respuesta:} }
%%%%%% Documento es inglés %%%%%%
%\newcommand{\question}[1]{\textbf{Question:} #1}
%\newcommand{\answer}{\textbf{Answer:} }
%===================================
\begin{document}
\noindent\textbf{Cinemática y dinámica de robots} \hfill \textbf{UDLAP}\\ % Nombre ddel curso
Primavera 2025 \hfill EDEI\\ %Periodo del curso
\begin{center}
    \mytitle{Actividad }    %Ejemplo Actividad 1.2: Transformaciones Espaciales
\end{center}
%===================================
\mysection{Nombre: ID:}
\begin{enumerate}[label=\Alph*)]
    %% Copie la siguiente sección tantas veces como preguntas tenga que responder
    %=========================================================================================================
    \item \question{}\\ %Copiar y pegar pregunta a responder
    \answer \\
    % Insertar figura
    \begin{center}
        \includegraphics[width=0.5\textwidth]{robot.jpg}
    \end{center}
    % Escriba aquí la respuesta
    %=========================================================================================================
    \item \question{}\\ %Copiar y pegar pregunta a responder
    \answer \\
\end{enumerate}

\end{document}
